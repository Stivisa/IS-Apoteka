
\section{Analiza sistema}

Informacioni sistem veleprodaje apoteke obuhvata nabavku, skladištenje i prodaju lekova. Pored toga tu je knjigovodstvo i marketing. Sve nadgledano od strane menadzerskog tima.

\subsection{Opis funkcionalnih celina}
Izdvojene celine sistema i opis njihovih funkcija.

\subsubsection{Prodaja}
Prodajom se bavi komercijalista, prima porudžbinu kupca na osnovu koje pravi ponudu. U slučaju da klijent prihvati našu ponudu, sledi ugovaranje prodaje i izrada izlazne fakture i otpremnice. Izlazna faktura se šalje kupcu i knjigovodstvu, otpremnica ide u magacin. 

\subsubsection{Nabavka}
Za proces nabavke isto je zadužen komercijalista. Njihov rad je u skladu sa instrukcijama menadžera. Pravi se trebovanje tj. porudžbina koja se šalje dobavljaču. Od dobavljača primamo ponudu, za čime sledi eventualno ugovaranje posla. Nakon ugovorene saradnje, dobavljač nam šalje ulaznu fakturu, koja se prosleđuje knjigovodstvu, a prijemnica sa robom pristiže u magacin.

\subsubsection{Magacin}
Skladištenjem se bavi magacin tj. magacioneri. Tu se vrši prijem robe po prijemnici. Odnosno izdavanje robe po otpremnici. Robu zajedno sa otpremnicom vozač dostavlja do kupca. Magacioner povremeno vrši popis lekova radi utvrđivanja viškova i manjkova od idealnog stanja po fakturama, razrešavanjem zatečenog stanja se bavi knjigovodstvo.  Potrebno je voditi računa o rokovima upotrebe lekova, davati prioritet  prodaji lekovima sa kraćim rokom. Zaduženje magacionera je takođe labeliranje i pozicioniranje robe. Sistem bi trebao da omogući grafički prikaz organizacije robe u magacinu, radi efikasnije pretrage robe.

\subsubsection{Knjigovodtsvo}

Knjigodstvo obradjuje izlazne i izlazne fakture, tačnije vrši isplatu ulaznih i provera uplate izlaznih faktura. Provera se vrši dobijanjem dnevnog izvoda prometa sredstava od banke. Knjigovođe takođe vrše obračun poreza,troškova i obračun zarada. Još jedno zaduženje sektora knjigovodstva bila bi briga o rezultatima popisa iz magacina, obrađuju viškove i manjkove.

\subsubsection{Menadžment}
Menadžment rukovodi kadrovima. Nadgleda i daje instrukcije odeljenjima 
prodaje,nabavke,knjigovodstva,marketinga.

\subsubsection{Administracija}
Administracija bavi podacima o klijentima tj. kupcima, i zaposlenima. Ovde su administrativne uloge dodeljenje komercijalistima, odnosno knjigovođama.

\subsubsection{Marketing}
Marketinški tim je zadužen za povećanje plasmana robe putem reklamnog materijala, koji se dostavlja kupcima. Neophodna je evidencija poslednjeg dostavljanja reklamnog materijala kupcu ,da ne bi došlo do bespotrebnog pretrpavanja kupca reklamnim materijalom ili pak zapostavljanje kupca.

\subsection{Uloge aktera u sistemu}
Navodimo konkretan spisak zaposlenih kao i njihova zaduženja. U konkretnoj firmi su vrlo verovatno potrebne modifikacije pojedinih uloga. Dok ovde predstavljamo raspodelu zaduženja kako smo mi zamislili da treba, što nije nužno najefikasnije u realnom sprovođenju sistema. 

\subsubsection{Komercijalista}
Može da radi u okviru odeljenja prodaje i nabavke, ili samo u jednom.
\begin{enumerate}
\item Prodaja
\begin{itemize}
\item Prima porudžbinu od kupca putem telefona,elektronske pošte ili putem sajta (sajt nije naša nadležnost, porudžbina bi se skladištila ne mestu u bazi koje nama odgovara)
\item Sastavlja i šalje ponudu kupcu. Ponuda može biti nekompletna u slučaju da nemamo tražene proizvode. 
\item Ugovara prodaju sa kupcem.
\item Izrađuje fakturu i otpremnicu. Fakturu prosleđuje kupcu i knjigovodstvu, otpremnicu magacinu.
\end{itemize}
\item Nabavka 
\begin{itemize}
\item Pravi se trebovanje na osnovu potreba vezanih za prodaju i procena menadžera.
\item Prima ponude dobavljača i ugovara kupovinu.
\item Prima fakturu od dobavljača, koju prosleđuje knjigovodstvu.
\end{itemize}
\end{enumerate}

\subsubsection{Knjigovođa}
\begin{itemize}
\item Obrada faktura, isplata ulaznih i provera uplata izlaznih faktura.
\item Evidencija o kupcima.
\item Obračun poreza.
\item Obračun zarada.
\item Obračun troškova(kancelarijski materijal, gorivo,...)
\item Evidentira i obrađuje viškove i manjkove dobije popisom iz magacina.
\end{itemize}

\subsubsection{Magacioner}
\begin{itemize}
\item Prijem robe po prijemnici.
\item Spremanje robe za transport po otpremnici.
\item Pazi oko rokova upotrebe lekova.
\item Labeliranje i pozicioniranje robe
\item Vrši popis tj. evidentira viškove i manjkove u magacinu.
\end{itemize}

\subsubsection{Vozač}
\begin{itemize}
\item Spremanje robe za transport po otpremnici.
\item Transport robe i predaja otpremnice kupcu.
\end{itemize}

\subsubsection{Menadžer}
Rukovodi kadrovima.
\begin{itemize}
\item Koordiniše sa nabavkom. Procenjuje potrebne količine za buduće poslovanje.
\item Koordiniše sa prodajom. Procenjuje kome je potrebno uključiti popust u ponudu.
\item Koordiniše sa knjigovodstvom.
\item Koordiniše sa marketingom.
\end{itemize}

\subsubsection{Marketing}
\begin{itemize}
\item Izrada reklamnog materijala
\item Dostava istog klijentima i budućim kupcima
\end{itemize}

\clearpage